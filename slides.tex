\documentclass[compress, aspectratio=169,usepdftitle=false]{beamer}
%\setbeameroption{show notes on second screen=left}
\usepackage[utf8]{inputenc}
\usepackage{braket}
\newcommand{\identity}[0]{\mathbf{1}}
\newcommand{\Op}[1]{\ensuremath{\mathsf{\hat{#1}}}}
\def\mat#1{\hat{#1}}
\def\half{ \frac{1}{2}}
\newcommand{\TildeOp}[1]{\ensuremath{\mathsf{\tilde{#1}}}}
\newcommand{\vectorize}{\operatorname{vec}}
\newcommand{\Abs}[1]{\left|#1\right|}
\newcommand{\AbsSq}[1]{\left|#1\right|^2}
\newcommand{\Norm}[1]{\left\lVert#1\right\rVert}
\newcommand{\NormSq}[1]{\Norm{#1}^2}
\newcommand{\tr}{\mathsf{tr}}
\newcommand{\Tr}{\mathsf{tr}}
\newcommand{\SU}{\ensuremath{\text{SU}}}
\newcommand{\ketbra}[2]{\ket{#1}\!\bra{#2}}
\newcommand{\mirror}{\text{mirror}}
\newcommand{\tgt}{\text{tgt}}
\newcommand{\pop}{\operatorname{pop}}
\newcommand{\dd}{\mathsf{d}}
\newcommand{\ii}{\mathsf{i}}
\newcommand{\Integers}{\mathbb{Z}}
\newcommand{\norm}{\operatorname{norm}}
\renewcommand{\Re}{\mathsf{Re}}
\renewcommand{\Im}{\mathsf{Im}}
\newcommand{\partdifquo}[2][{}]{\frac{\partial #1}{\partial #2}}
\newcommand{\Reals}{\mathbb{R}}
\newcommand{\Complex}{\mathbb{C}}
\newcommand{\Liouvillian}{\mathcal{L}}
\newcommand{\TimeOrder}{\mathcal{T}}
\newcommand{\SigmaX}{\Op{\sigma}_x}
\newcommand{\SigmaY}{\Op{\sigma}_y}
\newcommand{\SigmaZ}{\Op{\sigma}_z}
\newcommand{\SigmaPlus}{\Op{\sigma}_{\!+}}
\newcommand{\SigmaMinus}{\Op{\sigma}_{\!-}}
\usepackage{textcomp} % provides \textmu
\usepackage{tikz}
\usepackage{hyperref}
\usepackage{fontawesome}
\usetikzlibrary{shapes, arrows.meta, calc, decorations.pathmorphing, backgrounds, positioning}

\usepackage{amsmath, xparse, letltxmacro}
\LetLtxMacro{\oldunderbrace}{\underbrace}

\DeclareDocumentCommand{\underbrace}{d<> m e{_}}{%
  \IfValueTF{#1}{% IF <overlay-specification> given
    % using global onlside flag, cf. p82 beamer manual v3.59
    \oldunderbrace{#2\onslide<#1>}_{#3}\onslide%
  }{% ELSE
    \oldunderbrace{#2}_{#3}
  }%
}%

%% Notes on screenshots:
%
% - Make sure ``Reduce Transparency'' in the Accessibility settings (Display) is off
% - Size window to 1270 x 625 (2540 x 1250 retina)
%   - if no title on slide: increase height +41 to 666  (1332 retina)
% - Take screenshots at retina resolution
% - Terminal (iTerm) is at standard size +5 font size increases
% - JupyterLab in Arc – for slide without title:
%   - total window size 1364 x 800
%   - ``Simple Interface'' (View Menu)
%   - ``Presentation Mode'' (View Menu)
%   - No status bar (View Menu)


\input{arlwide_theme/theme.tex}

\hypersetup{%
  pdfauthor={Michael Goerz},
  pdftitle={Optimal Control Techniques for Quantum Interferometry},
  pdfsubject={Use of QuantumControl.jl in the context of quantum interferometry},
  pdfkeywords={quantum control; optimal control; quantum dynamics; interferometry; robustness; Julia},
}

\title{Optimal Control Techniques \\for Quantum Interferometry}
\author[\href{https://michaelgoerz.net}{https://michaelgoerz.net}]{Michael~H.~Goerz}
\institute[Army Research Lab]{DEVCOM Army Research Lab}
\date{IMSI Quantum Hardware Workshop, Chicago, October 30, 2024}

\begin{document}

{%  Title page
  \setbeamertemplate{footline}{}
  \frame{%
    \titlepage
    \note{\dots}
  }
}
\addtocounter{framenumber}{-1}

\begin{frame}{Quantum Interferometry}
  \begin{itemize}
    \item Ensemble of atoms \pause
    \item Create a superposition \pause
    \item Separate components \pause
    \item Accumulate a relative phase \pause
    \item Recombine \pause
    \item Imprint phase on measurement
  \end{itemize}
\end{frame}

\begin{frame}{Atomic Fountain Interferometer}
  \begin{textblock}{3}(1.0,1.5)
    \includegraphics<1,3->[height=6cm]{images/atomic_fountain.pdf}
  \end{textblock}
  \begin{textblock}{14}(1.0,2.5)
    \hfill\includegraphics<2>[trim=0 0 0 0,clip]{images/afioct/fig1.pdf}
  \end{textblock}
  \begin{textblock}{14}(1.0,1.5)
    \hfill\includegraphics<4>[trim=8.3cm 0 0 0,clip]{images/afioct/fig2.pdf}
  \end{textblock}
  \begin{textblock}{10}(5.0,7.5)
      \onslide<2,4->{%
        \hfill\footnotesize{--- Goerz, Kasevich, Malinovsky.  Atoms 11, 36 (2023)}
      }
  \end{textblock}
\end{frame}

\begin{frame}{Atomic Fountain Interferometer -- Robustness}
  \begin{textblock}{3}(1.0,1.5)
    \includegraphics[height=6cm]{images/atomic_fountain.pdf}
  \end{textblock}
  \begin{textblock}{13}(1.0,2.25)
    \hfill\includegraphics[trim=0 0 9.2cm 0,clip]{images/afioct/fig4.pdf}
  \end{textblock}
  \begin{textblock}{1}(14.0,2.25)
    \includegraphics[trim=12.6cm 0 0 0,clip]{images/afioct/fig4.pdf} % colorbar
  \end{textblock}
  \begin{textblock}{10}(5.0,7.5)
    \onslide<1->{%
      \hfill\footnotesize{--- Goerz, Kasevich, Malinovsky.  Atoms 11, 36 (2023)}
    }
  \end{textblock}
\end{frame}

\begin{frame}{Atomic Fountain Interferometer -- Rapid Adiabatic Passage (RAP)}
  \begin{textblock}{3}(1.0,1.5)
    \includegraphics<1-2>[height=6cm]{images/atomic_fountain.pdf}
  \end{textblock}
  \begin{textblock}{3}(12.2,5.2)
    \includegraphics<2->[trim=9.0cm 0 0 0,clip,height=3cm]{images/afioct/fig1.pdf}
  \end{textblock}
  \begin{textblock}{14}(1.0,1.5)
    \hfill\includegraphics<1-2>[trim=8.3cm 0 0 0,clip]{images/afioct/fig2.pdf}
  \end{textblock}
  \begin{textblock}{14}(1.0,1.5)
    \hfill\includegraphics<3>[trim=6.0cm 0 0 0,clip]{images/afioct/fig3.pdf}
  \end{textblock}
  \begin{textblock}{10}(5.0,7.5)
    \footnotesize{--- Goerz, Kasevich, Malinovsky.  Atoms 11, 36 (2023)}
  \end{textblock}
\end{frame}


\begin{frame}{Atomic Fountain Interferometer -- Robustness RAP}
  \begin{textblock}{3}(1.0,1.5)
    \includegraphics<1-2>[height=6cm]{images/atomic_fountain.pdf}
  \end{textblock}
  \begin{textblock}{13}(1.0,2.25)
    \hfill\includegraphics<1>[trim=0 0 9.2cm 0,clip]{images/afioct/fig4.pdf}
  \end{textblock}
  \begin{textblock}{13}(1.0,2.25)
    \hfill\includegraphics<2>[trim=0 0 5.2cm 0,clip]{images/afioct/fig4.pdf}
  \end{textblock}
  \begin{textblock}{1}(14.0,2.25)
    \includegraphics[trim=12.6cm 0 0 0,clip]{images/afioct/fig4.pdf} % colorbar
  \end{textblock}
  \begin{textblock}{10}(5.0,7.5)
    \onslide<1->{%
      \hfill\footnotesize{--- Goerz, Kasevich, Malinovsky.  Atoms 11, 36 (2023)}
    }
  \end{textblock}
\end{frame}

\begin{frame}{Atomic Fountain Interferometer -- Optimal Control (OCT)}
  \begin{textblock}{14}(1.0,1.5)
    \hfill\includegraphics<1>[trim=6.0cm 0 0 0,clip]{images/afioct/fig3.pdf}
  \end{textblock}
  \begin{textblock}{14}(1.0,2.0)
    \includegraphics<2>[trim=0 0 9.5cm 0, clip]{images/afioct/fig6.pdf}
  \end{textblock}
  \begin{textblock}{14}(1.0,2.0)
    \includegraphics<3>[trim=0 0 4.25cm 0, clip]{images/afioct/fig6.pdf}
  \end{textblock}
  \begin{textblock}{14}(1.0,2.0)
    \includegraphics<4>{images/afioct/fig6.pdf}
  \end{textblock}
  \begin{textblock}{14}(1.0,1.5)
    \hfill\includegraphics<5>[trim=6.0cm 0 0 0,clip]{images/afioct/fig3.pdf}
  \end{textblock}
  \begin{textblock}{14}(1.0,1.5)
    \hfill\includegraphics<6>[trim=0 0 0 0,clip]{images/afioct/fig7.pdf}
  \end{textblock}
  \begin{textblock}{10}(5.0,7.5)
    \onslide<1->{%
      \hfill\footnotesize{--- Goerz, Kasevich, Malinovsky.  Atoms 11, 36 (2023)}
    }
  \end{textblock}
\end{frame}


\begin{frame}{Atomic Fountain Interferometer -- Robustness OCT}
  \begin{textblock}{3}(1.0,1.5)
    \includegraphics<1>[height=6cm]{images/atomic_fountain.pdf}
  \end{textblock}
  \begin{textblock}{13}(1.0,2.25)
    \hfill\includegraphics<1>[trim=0 0 5.2cm 0,clip]{images/afioct/fig4.pdf}
  \end{textblock}
  \begin{textblock}{13}(1.0,2.25)
    \hfill\includegraphics<2>[trim=0 0 1.2cm 0,clip]{images/afioct/fig4.pdf}
  \end{textblock}
  \begin{textblock}{1}(14.0,2.25)
    \includegraphics[trim=12.6cm 0 0 0,clip]{images/afioct/fig4.pdf}
  \end{textblock}
  \begin{textblock}{10}(5.0,7.5)
    \onslide<1->{%
      \hfill\footnotesize{--- Goerz, Kasevich, Malinovsky.  Atoms 11, 36 (2023)}
    }
  \end{textblock}
\end{frame}


\begin{frame}{JuliaQuantumControl}
  \begin{textblock}{15.5}(0.25,1.00)
    \includegraphics<2>[width=\textwidth]{images/JuliaQuantumControl}
    % \includegraphics<3>[width=\textwidth]{images/JuliaQuantumControlPackages}
  \end{textblock}
\end{frame}


\begin{frame}
  \begin{center}
    {\Large \color{DarkRed} \bf Why Julia?}
    \vspace{1.5cm}
    \pause
    \begin{itemize}
      \item Performance \pause -- compiles to low-level machine code (matches Fortran) \pause
      \item Flexibility \pause -- multiple dispatch\pause, ecosystem\pause
      \item Expressiveness \pause -- clean syntax, unicode, notebook environment
    \end{itemize}
  \end{center}
\end{frame}


\begin{frame}{JuliaQuantumControl}
  \begin{textblock}{15.5}(0.25,1.00)
    \includegraphics<1->[width=\textwidth]{images/JuliaQuantumControl.png}
  \end{textblock}
  \begin{textblock}{14.5}(0.75,5.28)
    \onslide<2>{%
      \begin{block}{Design Principle}
        \begin{center}
          Maximum performance and composability through abstract interfaces
        \end{center}
      \end{block}
    }
  \end{textblock}
\end{frame}

\begin{frame}{Optimization Functional}
  \begin{textblock}{15.5}(0.25,1.00)
    \includegraphics[width=\textwidth]{images/functional.png}
  \end{textblock}
\end{frame}

\begin{frame}{Control Problem and Trajectories}
  \begin{textblock}{15.5}(0.25,1.00)
    \includegraphics<1>[width=\textwidth]{images/api_controlproblem.png}
    \includegraphics<2>[width=\textwidth]{images/api_controlproblem2.png}
    \includegraphics<3>[width=\textwidth]{images/api_trajectory.png}
    \includegraphics<4>[width=\textwidth]{images/api_trajectory2.png}
  \end{textblock}
\end{frame}

\begin{frame}{Dynamical Generators}
  \begin{textblock}{15.5}(0.25,1.00)
    \includegraphics<1>[width=\textwidth]{images/glossary_generator.png}
    \includegraphics<2>[width=\textwidth]{images/api_check_generator.png}
  \end{textblock}
\end{frame}

\begin{frame}{Optimization schemes}
  \begin{textblock}{15.5}(0.25,1.50)
    \includegraphics<2>[width=\textwidth]{images/schemes_comparison.pdf}
  \end{textblock}
  \begin{textblock}{10}(5.0,7.2)
    \onslide<2>{%
      \hfill\footnotesize{--- Goerz, Carrasco, Malinovsky.  Quantum 6, 871 (2022)}
    }
  \end{textblock}
\end{frame}


\begin{frame}{Propagators}
  \begin{textblock}{15.5}(0.25,1.00)
    \includegraphics<2>[width=\textwidth]{images/api_propagator.png}
    \includegraphics<3>[width=\textwidth]{images/api_propagator2.png}
    \includegraphics<4>[width=\textwidth]{images/differentialequations.png}
  \end{textblock}
\end{frame}


\begin{frame}{Rotating Tractor Interferometer}
  \begin{textblock}{13.5}(1.25,0.75)
    \begin{center}
      \includegraphics<2>{images/pinwheel}
    \end{center}
  \end{textblock}
  \begin{textblock}{15.5}(1.5,1.50)
    \only<3->{%
      \includegraphics{images/rottai_concept}
    }
  \end{textblock}
  \begin{textblock}{8.0}(6.5,1.75)
    \only<4->{%
      \begin{equation*}
        H_{\pm}(\theta, t) = -\frac{\hbar^2}{2M}\frac{\partial^2}{\partial \theta^2} + V_0 \cos\left(m (\theta + \phi_{\pm}(t) )\right)
      \end{equation*}
    }
  \end{textblock}
  \begin{textblock}{14.0}(1.0,6.25)
    \only<4->{%
      In co-moving frame:
      \vspace{-1.1cm}
      \begin{equation*}
          \hspace{2.5cm}
          \tilde{H}_{\pm} (t)= -\frac{\hbar^2}{2M}\frac{\partial^2}{\partial \theta^2} + V_0 \cos\left(m \theta\right) - i \hbar \omega_{\pm}(t) \frac{\partial}{\partial \theta}
      \end{equation*}
    }
  \end{textblock}
  \begin{textblock}{10}(5.0,8.2)
    \onslide<2->{%
      \hfill\footnotesize{--- Dash, Goerz \emph{et al.} AVS Quantum Sci. 6, 014407 (2023)}
    }
  \end{textblock}
\end{frame}


\begin{frame}{Project-Specific Data Structures}
  \begin{textblock}{15.5}(0.25,1.00)
    \includegraphics[width=\textwidth]{images/rottai_code}
  \end{textblock}
\end{frame}

\begin{frame}{Adiabatic Dynamics of Rotating TAI}
  \begin{textblock}{7.0}(0.5,1.00)
    \includegraphics<1-3>{images/animate_rottai/frame_000.pdf}
    \includegraphics<4>{images/animate_rottai/frame_100.pdf}
    \includegraphics<5>{images/animate_rottai/frame_200.pdf}
    \includegraphics<6->{images/animate_rottai/frame_300.pdf}
  \end{textblock}
  \begin{textblock}{7.0}(0.5,2.00)
    \onslide<2>{%
      \begin{center}
        {\color{DarkRed}$\pi/2$ pulse}
      \end{center}
    }
  \end{textblock}
  \begin{textblock}{7.0}(0.5,2.00)
    \onslide<9>{%
      \begin{center}
        {\color{DarkRed}inverse \\ $\pi/2$ pulse}
      \end{center}
    }
  \end{textblock}
  \begin{textblock}{7.0}(0.5,5.50)
    \onslide<3->{%
      \begin{equation*}
        \omega(t) = \begin{cases}
          \omega_{0} \sin^2\left(\frac{\pi t}{2 t_r}\right)         & 0 \leq t < t_r            \\
          \omega_{0}                                                & t_r \leq t < t_r + t_{\text{loop}}  \\
          \omega_{0} \cos^2\left(\frac{\pi t^\prime }{2t_r} \right) & T - t_r \leq t \leq T
        \end{cases}
      \end{equation*}
    }
  \end{textblock}
  \begin{textblock}{7.75}(8.0,2.00)
    \begin{center}
      \includegraphics<3-7>{images/adiabatic_dynamics_50πps_1}
      \includegraphics<8->{images/adiabatic_dynamics_50πps_2}
    \end{center}
  \end{textblock}
  \begin{textblock}{10}(5.0,8.2)
    \onslide<1->{%
      \hfill\footnotesize{--- Dash, Goerz \emph{et al.} AVS Quantum Sci. 6, 014407 (2023)}
    }
  \end{textblock}
\end{frame}

\begin{frame}{Interferometric Response of Rotating TAI}
  \begin{textblock}{15.5}(0.25,1.30)
    \begin{equation*}
      \Delta \Phi_S = \frac{4 m\Omega A}{\hbar}\,,
      \quad
      A
      = \frac{R^2}{2}
        \underbrace{\int_{0}^{T}\omega(t^\prime)dt^\prime}_{=\only<1-4>{n}\only<5>{2}\only<6>{10}\pi}
    \end{equation*}
  \end{textblock}
  \begin{textblock}{15.5}(0.25,3.00)
    \onslide<2->{%
      \begin{equation*}
        |c_{\pm}|^2
        = \frac{1}{2}
          \pm \frac{1}{2} \Re\left[{\color<3>{DarkRed}\eta} e^{-i \Delta\Phi}\right]
        \onslide<4->{%
          \qquad \rightarrow  \qquad
          |c_{-}|^2
          = \frac{1}{2} - \frac{\cos{\Delta\Phi}}{2} = \sin^2\left(\frac{\Delta\Phi}{2}\right)
        }
      \end{equation*}
    }
  \end{textblock}
  \begin{textblock}{4}(1.85,4.00)
    \onslide<3-4>{%
      \begin{equation*}
        \color{DarkRed}
        \eta = \braket{\Psi_{-}(\theta, T)|\Psi_{+}(\theta, T)}
        = 1 \quad \text{if adiabatic}
      \end{equation*}
    }
  \end{textblock}
  \begin{textblock}{15.5}(0.25,4.2)
    \onslide<5->{%
      \begin{center}
        \includegraphics<5>{images/cn_sim_results_1.pdf}
        \includegraphics<6>{images/cn_sim_results_2.pdf}
      \end{center}
    }
  \end{textblock}
  \begin{textblock}{10}(5.0,8.2)
    \onslide<5->{%
      \hfill\footnotesize{--- Dash, Goerz \emph{et al.} AVS Quantum Sci. 6, 014407 (2023)}
    }
  \end{textblock}
\end{frame}


\begin{frame}{Non-Adiabatic Dynamics of Rotating TAI}
  \begin{textblock}{7.75}(0.25,2.00)
    \includegraphics<1>{images/fidelity_map_1}
    \includegraphics<2-8>{images/fidelity_map_2}
  \end{textblock}
  \begin{textblock}{7.0}(8.5,2.25)
    \onslide<1-2>{%
      \begin{equation*}
        \omega(t) = \begin{cases}
          \omega_{0} \sin^2\left(\frac{\pi t}{2 t_r}\right)         & 0 \leq t < t_r            \\
          {\color{gray}\omega_{0}}                                  & {\color{gray}t_r \leq t < t_r + t_{\text{loop}}}  \\
          {\color{gray}\omega_{0} \cos^2\left(\frac{\pi t^\prime }{2t_r} \right)} & {\color{gray}T - t_r \leq t \leq T}
        \end{cases}
      \end{equation*}
      \vspace{2mm}
      \par
      $\ket{\Psi_{\text{tgt}}} = $ ground state of moving potential
    }
  \end{textblock}
  \begin{textblock}{7.75}(8.00,1.00)
    \includegraphics<3>{images/guess_dynamics_1.pdf}
    \includegraphics<4>{images/guess_dynamics_2.pdf}
    \includegraphics<5>{images/guess_dynamics_3.pdf}
    \includegraphics<6>{images/guess_dynamics_4.pdf}
    \includegraphics<7>{images/guess_dynamics_5.pdf}
    \includegraphics<8->{images/guess_dynamics_6.pdf}
  \end{textblock}
  \begin{textblock}{7.75}(0.25,2.00)
    \includegraphics<9-11>{images/guess_sagnac_1.pdf}
    \includegraphics<12>{images/guess_sagnac_2.pdf}
  \end{textblock}
  \begin{textblock}{7.75}(0.25,5.75)
    \onslide<9->{%
      \begin{equation*}
        \Delta \Phi_S = \frac{4 m\Omega A}{\hbar}\,,
        \quad
        A = \frac{R^2}{2} \cdot 10\pi
      \end{equation*}
    }
  \end{textblock}
  \begin{textblock}{7.75}(0.25,7.00)
    \only<9>{%
      \begin{equation*}
        |c_{-}|^2
        = \frac{1}{2} - \frac{\cos{\Delta\Phi}}{2} = \sin^2\left(\frac{\Delta\Phi}{2}\right)
      \end{equation*}
    }
    \only<10->{%
      \begin{equation*}
        |c_{-}|^2
        = \frac{1}{2} - \frac{1}{2} \Re\left[{\color<11>{DarkRed}\eta} e^{-i \Delta\Phi}\right]
      \end{equation*}
    }
  \end{textblock}
  \begin{textblock}{10}(5.0,8.2)
    \onslide<1-6>{%
      \hfill\footnotesize{--- Dash, Goerz \emph{et al.} AVS Quantum Sci. 6, 014407 (2023)}
    }
  \end{textblock}
\end{frame}


\begin{frame}{Rotating TAI Control Problem}
  \begin{textblock}{15.0}(0.5,1.20)
    \begin{center}
      \begin{equation*}
        \omega(t) = \begin{cases}
          \omega_{\text{opt}}(t)  & 0 \leq t < t_r            \\
          \omega_{0}              & t_r \leq t < t_r + t_{\text{loop}}  \\
          \omega_{\text{opt}}(t') & T - t_r \leq t \leq T
        \end{cases}
      \end{equation*}
      \par
      \vspace{8mm}
      Find $\omega_{\text{opt}}(t)$ for short $t_r$ so that
      \begin{equation*}
        \color{DarkRed}
        \Psi(\theta, t=0) \rightarrow \Psi_{\text{tgt}}(\theta, t=t_r)
      \end{equation*}
      where $\ket{\Psi_{\text{tgt}}} = $ ground state of moving potential
    \end{center}
  \end{textblock}
\end{frame}


\begin{frame}{Optimization with QuantumControl.jl}
  \begin{textblock}{15.5}(0.25,1.00)
    \includegraphics<1-2>[width=\textwidth]{images/optimization_screenshot1}
    \includegraphics<3>[width=\textwidth]{images/optimization_screenshot2}
  \end{textblock}
  \begin{textblock}{12.8}(2.0,3.70)
    \onslide<2>{%
      \begin{block}{Guided Control}
        \vspace{2mm}
        \begin{equation*}
          \omega_{\text{opt}}(t) = \omega(t) + S(t)\delta\omega(t)
        \end{equation*}
        \vspace{2mm}
      \end{block}
    }
  \end{textblock}
\end{frame}


\begin{frame}{Optimized Dynamics or Rotating TAI}
  \begin{textblock}{7.75}(0.25,1.00)
    \includegraphics<1-6>{images/guess_dynamics.pdf}
  \end{textblock}
  \begin{textblock}{7.75}(8.00,1.00)
    \includegraphics<2>{images/opt_dynamics_1.pdf}
    \includegraphics<3>{images/opt_dynamics_2.pdf}
    \includegraphics<4>{images/opt_dynamics_3.pdf}
    \includegraphics<5>{images/opt_dynamics_5.pdf}
    \includegraphics<6->{images/opt_dynamics_6.pdf}
  \end{textblock}
  \begin{textblock}{7.75}(0.25,2.00)
   \includegraphics<7>{images/opt_sagnac_1.pdf}
   \includegraphics<8->{images/opt_sagnac_2.pdf}
  \end{textblock}
  \begin{textblock}{7.75}(0.25,5.75)
    \onslide<7->{%
      \begin{equation*}
        \Delta \Phi_S = \frac{4 m\Omega A}{\hbar}\,,
        \quad
        A = \frac{R^2}{2} \cdot 10\pi
      \end{equation*}
    }
  \end{textblock}
  \begin{textblock}{7.75}(0.25,7.00)
   \only<7>{%
     \begin{equation*}
       |c_{-}|^2
       = \frac{1}{2} - \frac{1}{2} \Re\left[\eta e^{-i \Delta\Phi}\right]
     \end{equation*}
   }
   \only<8->{%
     \begin{equation*}
       |c_{-}|^2
       = \frac{1}{2} - \frac{\cos{\Delta\Phi}}{2} = \sin^2\left(\frac{\Delta\Phi}{2}\right)
     \end{equation*}
   }
  \end{textblock}
\end{frame}


\begin{frame}{Nuclear Spin Gyroscope}
  \begin{textblock}{15}(0.5,1.0)
    \onslide<2>{%
      \begin{center}
        \includegraphics[width=\textwidth]{images/JarmolaSA2021_Fig2}
      \end{center}
      \hfill {\footnotesize --- Adapted from Fig 2 of Jarmola \textit{et. al.} Sci. Adv. 7, eabl3840 (2021)}
    }
  \end{textblock}
\end{frame}


\begin{frame}{NV Center Hamiltonian with Crosstalk}
  \begin{textblock}{6}(0.5,2.00)
    %\documentclass[aps, pra, onecolumn, superscriptaddress, floatfix]{revtex4-2}
%\usepackage[utf8]{inputenc}
%\usepackage{amsmath}
%\usepackage{mathpazo}
%\usepackage{braket}
%\usepackage[hidelinks]{hyperref}
%\usepackage{graphicx}
%
%
%\usepackage{tikz,pgflibraryshapes}
%\usetikzlibrary{calc}
%\usetikzlibrary{decorations.pathmorphing}
%\usetikzlibrary{arrows.meta}
%\usetikzlibrary{positioning}
%
%\usepackage[psfixbb,graphics,tightpage,active]{preview}
%
%\PreviewEnvironment{tikzpicture}
%
\tikzset{arrow/.style={-{Latex[length=3pt]}}}
\tikzset{doublearrow/.style={{Latex[length=3pt]}-{Latex[length=3pt]}}}
%
%\begin{document}
\begin{tikzpicture}
  \definecolor{red}{RGB}{192,0,0}
  \definecolor{blue}{RGB}{3,110,189}
  \footnotesize
  % \node[right] at (-0.7, 3.5)  {(a)};
  % \node[right] at (3.2, 3.5)  {(b)};
  % \begin{scope}[xscale=0.25,yscale=0.28] % Level Diagram
  \begin{scope}[xscale=0.4,yscale=0.4] % Level Diagram
    \def\LevelMinusOne{3}
    \def\LevelZero{10}
    \def\DeltaP{2}
    \def\DeltaS{4}
    \def\Width{10}
    \def\xOmegaOne{0.05}  % x coordinate for ω₁ line, as portion of \Width
    \def\xOmegaP{0.275}  % x coordinate for ωₚ line, as portion of \Width
    \def\xOmegaS{0.725}  % x coordinate for ωₛ line, as portion of \Width
    \pgfmathsetmacro{\SinglePhotonDetuning}{0.5*(\DeltaP + \DeltaS)}
    % Energy Levels
    \draw[thick] (0, 0) node[left]{$\ket{+1}$} -- +(\Width, 0);
    \draw[thick] (0.4*\Width, \LevelMinusOne) -- ++(0.6*\Width, 0) node[right]{$\ket{-1}$};
    \draw[thick] (0, \LevelZero) node[left]{$\ket{0}$} -- +(\Width, 0);
    % Detuning Levels
    \draw[dashed] (0.15 * \Width, \LevelZero - \DeltaP) -- +(0.35 * \Width, 0);
    \draw[dashed] (0.35 * \Width, \LevelZero - \DeltaS) -- +(0.5 * \Width, 0);
    \draw[dashed] (0.475 * \Width, \LevelZero - \SinglePhotonDetuning) -- +(0.2 * \Width, 0);
    %  Arrows
    \draw[arrow, red, thick] (\xOmegaP * \Width, 0) -- +(0, \LevelZero - \DeltaP) node[midway, left, red]{$\omega_p$};
    \draw[arrow] (\xOmegaOne * \Width, 0) -- +(0, \LevelZero) node[midway, left]{$\omega_1$};
    \draw[arrow] (\xOmegaP * \Width, \LevelZero) -- +(0, -\DeltaP) node[midway, left]{$\Delta_p$};
    \draw[arrow, blue, thick] (\xOmegaS * \Width, \LevelMinusOne) -- +(0, \LevelZero - \DeltaS - \LevelMinusOne) node[midway, right, blue]{$\omega_s$};
    \draw[arrow] (\xOmegaS * \Width, \LevelZero) -- +(0, -\DeltaS) node[midway, right]{$\Delta_s$};
    \draw[arrow] (0.95 * \Width, \LevelMinusOne) -- +(0, \LevelZero - \LevelMinusOne) node[midway, right]{$\omega_2$};
    \draw[arrow] (0.425 * \Width, \LevelZero - \DeltaS) -- +(0, \DeltaS - \DeltaP) node[midway, left]{$\delta$};
    \draw[arrow] (0.575 * \Width, \LevelZero) -- +(0, -0.5 * \DeltaS - 0.5 * \DeltaP) node[pos=0.28, left]{$\Delta$};
    \draw[arrow] (0.575 * \Width, 0) -- +(0, \LevelMinusOne) node[midway, right]{$\omega_{12}$};
  \end{scope}
  %\begin{scope}[xscale=0.7, yscale=1.0, xshift=5.5cm, yshift=0.55cm] % pulse sequence
  %  \def\pulseLength{1.5}
  %  \def\tauLength{1.5}
  %  \def\extraBeforeAfter{0.5}
  %  \def\boxPad{0.2}
  %  \pgfmathsetmacro{\timeLength}{2*\extraBeforeAfter + 2*\pulseLength + \tauLength}
  %  \draw[blue, thick] (-\extraBeforeAfter, 0) -- +(\timeLength, 0);
  %  \draw[red, thick] (-\extraBeforeAfter, 1) -- +(\timeLength, 0);
  %  \draw[dashed] (0,-0.5) node[below]{0} -- +(0, 3);
  %  \draw[dashed] (\pulseLength,-0.5) node[below]{$T$} -- +(0, 3);
  %  \draw[dashed] (\pulseLength + \tauLength, -0.5) node[below]{$T+\tau$} -- +(0, 3);
  %  \draw[dashed] (2 * \pulseLength + \tauLength,-0.5) node[below]{$2T + \tau$} -- +(0, 3);
  %  \draw[fill=blue] (\boxPad, 0) rectangle +(\pulseLength - 2 * \boxPad, 1-\boxPad) node[pos=.5,white] {$\Omega_s$};;
  %  \draw[fill=blue] (\pulseLength + \tauLength + \boxPad, 0) rectangle +(\pulseLength - 2 * \boxPad, 1-\boxPad) node[pos=.5,white] {$\Omega_s$};;
  %  \draw[fill=red] (\boxPad, 1) rectangle +(\pulseLength - 2 * \boxPad, 1-\boxPad) node[pos=.5,white] {$\Omega_p$};;
  %  \draw[fill=red] (\pulseLength + \tauLength + \boxPad, 1) rectangle +(\pulseLength - 2 * \boxPad, 1-\boxPad) node[pos=.5,white] {$\Omega_p$};;
  %  \draw[semithick, rounded corners] (0, -\boxPad) rectangle +(\pulseLength, 2 + \boxPad);
  %  \draw[semithick, rounded corners] (\pulseLength + \tauLength, -\boxPad) rectangle +(\pulseLength, 2 + \boxPad);
  %  \node at (0.5 * \pulseLength, 2.25) {$\pi/2$};
  %  \node at (\pulseLength + \tauLength + 0.5 * \pulseLength, 2.25) {$\pi/2$};
  %  \draw[doublearrow] (\pulseLength, 2.25) -- +(\tauLength, 0) node[midway, below]{$\tau$};
  %\end{scope}
\end{tikzpicture}
%\end{document}

    % \includegraphics<1>[trim=0 0 4cm 5mm,clip]{images/nvcenter_system_diagram.pdf}
  \end{textblock}
  \begin{textblock}{8}(7.0,2.00)
    \only<1>{%
      \begin{equation*}
        \Op{H} = \begin{pmatrix}
              0     & \Omega(t) &    0                       \\
          \Omega(t) & \omega_1  & \Omega(t)                  \\
              0     & \Omega(t)  & \omega_1 - \omega_2
        \end{pmatrix}
      \end{equation*}
      \begin{align*}
          \Omega(t) &= \Omega_p(t) \cos\left(\omega_p t + \phi_p(t) + \phi_p^{(i)}\right)  \\
                    &\quad + \Omega_s(t) \cos\left(\omega_s t + \phi_s(t) + \phi_s^{(i)}\right)
      \end{align*}
    }
    \only<2>{%
      \begin{equation*}
        \Op{H}_{\text{RWA}} = \begin{pmatrix}
            -\frac{\delta}{2}  &   \Omega_{1,0}(t)   &    0              \\
            \Omega_{1,0}^*(t)  &      \Delta         & \Omega_{0,-1}(t)  \\
                    0          & \Omega_{0,-1}^*(t)  & \frac{\delta}{2}
        \end{pmatrix}
      \end{equation*}
      \begin{align*}
        \Omega_{1,0}(t) &= \frac{\Omega_p(t)}{2} e^{i \phi_p^{(i)}} + \frac{\Omega_s(t)}{2} e^{i\phi_s^{(i)}} e^{-i \omega_{ps} t} \\
        \Omega_{0,-1}(t) &= \frac{\Omega_s^*(t)}{2} e^{-i \phi_p^{(i)}} + \frac{\Omega_p^*(t)}{2} e^{-i\phi_s^{(i)}} e^{-i \omega_{ps} t}
      \end{align*}
    }
  \end{textblock}
  \begin{textblock}{15.5}(0.25,1.00)
    \includegraphics<3>[width=\textwidth]{images/ramsey_ham_code.png}
  \end{textblock}
\end{frame}


\begin{frame}{Parameterized Pulses}
  \begin{textblock}{15.5}(0.25,1.00)
    \includegraphics<2>[width=\textwidth]{images/nvramsey_nb_parameterziation.png}
    \includegraphics<3>[width=\textwidth]{images/pulse_parameterization.png}
  \end{textblock}
\end{frame}


\begin{frame}{Population Response Signal}
  \begin{textblock}{15}(0.5,1.0)
    \begin{center}
      \includegraphics<2>{images/ramsey_1}
      \includegraphics<3>{images/ramsey_2}
      \includegraphics<4>{images/ramsey_3}
      \includegraphics<5->{images/ramsey_4}
    \end{center}
  \end{textblock}
  \begin{textblock}{15}(0.5,5.0)
    \begin{center}
      \includegraphics<2-5>[height=3.5cm]{images/JarmolaSA2021_Fig2}
    \end{center}
  \end{textblock}
  \begin{textblock}{13}(1.5,5.5)
    \onslide<6->{%
      \begin{center}
        \begin{block}{}
          \begin{equation*}
            J_T(\{\ket{\Psi_{\mu,\tau}(T)}\})
            = \sum_{\mu} \left\vert
            \text{FFT}([P_0({\color<7>{DarkRed}\tau}; {\color<7>{DarkRed}\mu})]) - \text{FFT}([P_0({\tau}; {\mu}=1)])
              \right\vert
          \end{equation*}
          Make spectrum for any $\mu$ look like spectrum for $\mu = 1$
        \end{block}
      \end{center}
    }
  \end{textblock}
\end{frame}


\begin{frame}{NV Center Optimization Problem}
  \begin{textblock}{15.5}(0.25,1.00)
    \includegraphics<2,6>[width=\textwidth]{images/nvramsey_nb_problem.png}
    \includegraphics<1>[width=\textwidth]{images/nvramsey_nb_problem1.png}
    \includegraphics<3-4>[width=\textwidth]{images/nvramsey_nb_problem2.png}
    \includegraphics<5>[width=\textwidth]{images/nvramsey_nb_problem3.png}
    \includegraphics<7->[width=\textwidth]{images/nvramsey_nb_problem4.png}
  \end{textblock}
  \begin{textblock}{13}(1.5,5.5)
    \onslide<1-3,5-6>{%
      \begin{center}
        \begin{block}{}
          \begin{equation*}
            J_T(\{\ket{\Psi_{\mu,\tau}(T)}\})
            = \sum_{\mu} \left\vert
            {\color<6>{DarkRed}\text{FFT}}([P_0({\color<1-5>{DarkRed}\tau}; {\color<1>{DarkRed}\mu})]) - {\color<6>{DarkRed}\text{FFT}}([P_0({\color<2-5>{DarkRed}\tau}; {\mu}=1)])
              \right\vert
          \end{equation*}
          Make spectrum for any $\mu$ look like spectrum for $\mu = 1$
        \end{block}
      \end{center}
    }
  \end{textblock}
  \begin{textblock}{13}(1.5,5.5)
    \onslide<4>{%
      \begin{center}
        \begin{block}{}
          \begin{equation*}
            \Omega_{1,0}(t) = \frac{\Omega_p(t)}{2} e^{i \color{DarkRed}\phi_p^{(i)}} + \frac{\Omega_s(t)}{2} e^{i \color{DarkRed}\phi_s^{(i)}} e^{-i \omega_{ps} t}
          \end{equation*}
          Absorb phase difference in RF pulse
        \end{block}
      \end{center}
    }
  \end{textblock}
  \begin{textblock}{13}(1.5,2.05)
    \onslide<7->{%
      \begin{center}
        \begin{block}{}
          \begin{equation*}
            J_T(\{\ket{\Psi_{\mu,\tau}(T)}\})
            = \sum_{\mu} \left\vert
            {\color{DarkRed}\text{FFT}}([P_0({\tau}; {\mu})]) - {\color{DarkRed}\text{FFT}}([P_0({\tau}; {\mu}=1)])
              \right\vert
          \end{equation*}
          Make spectrum for any $\mu$ look like spectrum for $\mu = 1$
        \end{block}
      \end{center}
    }
  \end{textblock}
\end{frame}

\begin{frame}{Semi-automatic differentiation}
  \begin{textblock}{13}(1.5,2.05)
    \onslide<1>{%
      \begin{center}
        \begin{block}{}
          \begin{equation*}
            J_T(\{\ket{\Psi_{\mu,\tau}(T)}\})
            = \sum_{\mu} \left\vert
            {\color{DarkRed}\text{FFT}}([P_0({\tau}; {\mu})]) - {\color{DarkRed}\text{FFT}}([P_0({\tau}; {\mu}=1)])
              \right\vert
          \end{equation*}
          Make spectrum for any $\mu$ look like spectrum for $\mu = 1$
        \end{block}
      \end{center}
    }
  \end{textblock}
  \begin{textblock}{7}(8.5,2.25)
    \onslide<3>{%
      Automatic Differentiation:\par
      evaluate $J_T$ inside AD framework
    }
  \end{textblock}
  \begin{textblock}{14}(1.0,4.0)
    \onslide<4,5>{%
      \begin{block}{Semi-AD}
        Use a chain rule to split the gradient into
        \begin{itemize}
          \item a numerically expensive but analytic part
          \item a non-analytic but computationally cheap part
      \end{itemize}
      \end{block}
    }
  \end{textblock}
  \begin{textblock}{7}(1.0,2.0)
    \onslide<2->{%
      \begin{equation*}
        \begin{split}
          \nabla J_T
          &= \frac{\partial J_T\only<5->{(\{\Psi_k(T)\})}}{\partial \epsilon_{nl}}
          \\
          \onslide<6->{%
          &= 2 \Re \Bigg[
            \sum_k
              {\color<6>{white}\underbrace{\color{black}\frac{\partial J_T}{\partial \ket{\Psi_k(T)}}}_{\equiv \bra{\chi_k}}}
              \frac{\partial \ket{\Psi_k(T)}}{\partial \epsilon_{nl}}
            \Bigg]
          }
          \\
          \onslide<8->{%
          &= 2 \Re \Bigg[
            \sum_k
              \frac{\partial}{\partial \epsilon_{nl}}
              {\braket{\chi_k(T) | \Psi_k(T)}}
            \Bigg]
          }
        \end{split}
      \end{equation*}
    }
  \end{textblock}
  \begin{textblock}{7}(8.5,2.5)
    \includegraphics<9->[width=\textwidth]{images/grape_scheme}
  \end{textblock}
  \begin{textblock}{15}(1.0,7.5)
    \onslide<4->{%
      \footnotesize{--- Goerz, Carrasco, Malinovsky.  Quantum 6, 871 (2022)}
    }
  \end{textblock}
\end{frame}


\begin{frame}{NV Center Optimized Pulses}
  \begin{textblock}{15.5}(0.25,1.00)
    \includegraphics<2>[width=\textwidth]{images/nvramsey_nb_solution1.png}
    \includegraphics<3>[width=\textwidth]{images/nvramsey_nb_solution2.png}
  \end{textblock}
\end{frame}


\begin{frame}{Optimized Signal Spectrum}
  \begin{textblock}{15}(0.5,1.0)
    \begin{center}
      \includegraphics<1>{images/ramsey_4}
      \includegraphics<2>{images/ramsey_5}
      \includegraphics<3>{images/ramsey_6}
      \includegraphics<4>{images/ramsey_7}
      \includegraphics<5>{images/ramsey_8}
    \end{center}
  \end{textblock}
  \begin{textblock}{13}(1.5,5.5)
    \onslide<1>{%
      \begin{center}
        \begin{block}{}
          \begin{equation*}
            J_T(\{\ket{\Psi_{\mu,\tau}(T)}\})
            = \sum_{\mu} \left\vert
                \text{FFT}([P_0(\tau; \mu)]) - \text{FFT}([P_0(\tau; \mu=1)])
              \right\vert
          \end{equation*}
          Make spectrum for any $\mu$ look like spectrum for $\mu = 1$
        \end{block}
      \end{center}
    }
  \end{textblock}
\end{frame}

\begin{frame}{Conclusion}
  \begin{textblock}{13}(1.5,1.5)
    \onslide<1->{%
      \begin{itemize}
        \item<2-> Quantum Interferometry Implementations
          \begin{itemize}
            \item<3-> Atomic Fountain Interferometer: robust momentum space transfer
            \item<4-> Rotating Tractor Interferometer: non-adiabatic phase space transport
            \item<5-> Nuclear Spin Gyroscope: ``double quantum'' control, spectral optimization
          \end{itemize}
      \end{itemize}
    }
  \end{textblock}
  \begin{textblock}{14}(1.0,3.75)
    \includegraphics<6>[width=\textwidth]{images/attribution.pdf}
  \end{textblock}
  \begin{textblock}{13}(1.5,4.0)
    \onslide<7->{%
      \begin{itemize}
        \item<7-> QuantumControl.jl Framework
          \begin{itemize}
            \item<8-> Define control problems in terms of ``trajectories''
            \item<9-> Define dynamics in terms of ``generators'' and stateful ``propagators''
            \item<10-> Separate ``control amplitudes`` from actual ``controls''
            \item<11-> Efficient project-specific data structures through multiple dispatch
          \end{itemize}
      \end{itemize}
    }
  \end{textblock}
  \begin{textblock}{13}(1.5,7.0)
    \onslide<12>{%
      \begin{center}
        {\color{DarkRed}
        \Large Thank You!
        }
      \end{center}
    }
  \end{textblock}
\end{frame}


%%%%%%%%%%%%%%%%%%%%%%%%%%%%%%%%%%%%%%%%%%%%%%%%%%%%%%%%%%%%
\appendix
\backupbegin

\begin{frame} \end{frame}

\backupend

\end{document}
